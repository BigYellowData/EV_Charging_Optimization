\documentclass[12pt,a4paper]{article}
\usepackage[utf8]{inputenc}
\usepackage[french]{babel}
\usepackage[T1]{fontenc}
\usepackage{amsmath}
\usepackage{amsfonts}
\usepackage{amssymb}
\usepackage{graphicx}
\usepackage{hyperref}
\usepackage{booktabs}
\usepackage{xcolor}
\usepackage{geometry}
\usepackage{listings}
\usepackage{float}

\geometry{margin=2.5cm}

\title{\textbf{Optimisation Multi-Objectifs de la Charge\\de Véhicules Électriques}\\
\vspace{0.5cm}
\large Métaheuristique MODE appliquée aux données réelles Caltech}

\author{[Votre Nom]}
\date{17 Janvier 2026}

\begin{document}

\maketitle

\tableofcontents
\newpage

\section{Le Problème}

\subsection{Contexte et Référence}

Ce travail s'inspire de la modélisation mathématique de Qian et al. (2023) \cite{qian2023federated} sur le contrôle de charge de véhicules électriques avec V2G/G2V.

\textbf{Notre contribution :} Nous avons implémenté la \textbf{même formulation mathématique} mais avec une métaheuristique différente :
\begin{itemize}
    \item \textbf{Article original} : FedSAC (Federated Soft Actor-Critic) - apprentissage par renforcement
    \item \textbf{Notre approche} : MODE (Multi-Objective Differential Evolution) - algorithme évolutionnaire
\end{itemize}

Cette adaptation nous permet d'obtenir un \textbf{front de Pareto complet} avec 100 solutions non-dominées, offrant une flexibilité de choix au gestionnaire du réseau.

\subsection{Qu'est-ce qu'on veut faire ?}

Optimiser la charge de \textbf{30 véhicules électriques} en considérant \textbf{3 objectifs contradictoires} :

\begin{enumerate}
    \item \textbf{Minimiser le coût} d'électricité (ou même générer des revenus avec V2G)
    \item \textbf{Minimiser l'insatisfaction} des utilisateurs (atteindre le SoC cible)
    \item \textbf{Minimiser le pic de puissance} (impact sur le réseau électrique)
\end{enumerate}

\subsection{Pourquoi c'est compliqué ?}

Ces trois objectifs sont \textbf{contradictoires} :
\begin{itemize}
    \item Charger vite pour satisfaire $\rightarrow$ Coûte cher + Crée des pics
    \item Charger peu cher (la nuit) $\rightarrow$ Insatisfait ceux qui partent tôt
    \item Décharger pour gagner de l'argent (V2G) $\rightarrow$ Crée des pics
\end{itemize}

\textbf{Conclusion :} Il n'existe pas UNE solution optimale, mais un \textbf{ensemble de solutions} (front de Pareto).

\section{Les Données}

\subsection{Source}
\begin{itemize}
    \item \textbf{Caltech ACN-Data} : Données réelles de charge de VE
    \item \textbf{Date :} 15 juillet 2019 (un lundi normal)
    \item \textbf{Véhicules :} 30 sessions de charge réelles
\end{itemize}

\subsection{Tarif électricité (TOU - Time of Use)}

Le prix varie selon l'heure :
\begin{itemize}
    \item \textbf{Heures creuses} (0h-6h, 22h-24h) : 0.12 \$/kWh
    \item \textbf{Heures moyennes} (6h-16h) : 0.18 \$/kWh
    \item \textbf{Heures de pointe} (16h-22h) : 0.30 \$/kWh ($\times$2.5 plus cher !)
\end{itemize}

\textbf{Impact :} L'algorithme va naturellement privilégier la charge la nuit et la décharge (V2G) aux heures de pointe.

\section{Le Modèle Mathématique}

\subsection{Variable de décision}

\textbf{$P_{i,t}$} = Puissance de charge du véhicule $i$ à l'heure $t$ (en kW)

\begin{itemize}
    \item $i$ : véhicule (1 à 30)
    \item $t$ : heure (0 à 23)
    \item \textbf{Total :} $30 \times 24 = \textbf{720 variables}$
\end{itemize}

\textbf{Valeurs possibles :}
\begin{itemize}
    \item $P > 0$ : Charge (réseau $\rightarrow$ batterie)
    \item $P < 0$ : Décharge V2G (batterie $\rightarrow$ réseau)
    \item $P = 0$ : Rien
\end{itemize}

\textbf{Limites :} $-6 \text{ kW} \leq P \leq +30 \text{ kW}$

\subsection{Les 3 objectifs à minimiser}

\subsubsection{Objectif 1 : Coût (€)}

\begin{equation}
f_1(P) = \sum_{t=0}^{23} \left[ P_{\text{total}}(t) \times \text{Tarif}(t) \times \Delta t \right]
\end{equation}

où $P_{\text{total}}(t) = \sum_{i=1}^{30} P_{i,t}$

\begin{itemize}
    \item Si négatif $\rightarrow$ Profit (V2G)
    \item Si positif $\rightarrow$ Coût
\end{itemize}

\subsubsection{Objectif 2 : Insatisfaction}

\begin{equation}
f_2(P) = \sum_{i=1}^{30} \max(0, \text{SoC}_{\text{cible},i} - \text{SoC}_{\text{final},i})
\end{equation}

Mesure l'écart entre ce que veut l'utilisateur et ce qu'il obtient.

\subsubsection{Objectif 3 : Pic de puissance (kW)}

\begin{equation}
f_3(P) = \max_{t \in [0,23]} |P_{\text{total}}(t)|
\end{equation}

Impact sur le réseau électrique.

\subsection{Contraintes}

\begin{enumerate}
    \item \textbf{SoC batterie :} $0 \leq \text{SoC}_{i,t} \leq 1$ $\forall i, \forall t$
    \item \textbf{Puissance site :} $|P_{\text{total}}(t)| \leq 60$ kW $\forall t$
    \item \textbf{Disponibilité :} $P_{i,t} = 0$ si véhicule non connecté
\end{enumerate}

\section{L'Algorithme MODE}

\subsection{C'est quoi MODE ?}

\textbf{MODE} = Multi-Objective Differential Evolution

C'est un algorithme \textbf{évolutionnaire} : il fait évoluer une population de solutions comme dans la nature.

\textbf{Principe :}
\begin{enumerate}
    \item \textbf{Départ :} 100 solutions aléatoires
    \item \textbf{Itération} (1500 fois) :
    \begin{itemize}
        \item Créer de nouvelles solutions par mutation/croisement
        \item Garder les meilleures (celles non-dominées)
    \end{itemize}
    \item \textbf{Résultat :} Front de Pareto = ensemble des meilleures solutions
\end{enumerate}

\subsection{Paramètres utilisés}

\begin{table}[H]
\centering
\begin{tabular}{lcc}
\toprule
\textbf{Paramètre} & \textbf{Valeur} & \textbf{Signification} \\
\midrule
Population & 100 & Nombre de solutions par génération \\
Générations & 1500 & Nombre d'itérations \\
F (mutation) & 0.5 & Ampleur des variations \\
CR (croisement) & 0.9 & Fréquence d'échange de gènes \\
Variant & DE/rand/1/bin & Schéma de mutation \\
Seed & 1 & Pour la reproductibilité \\
\bottomrule
\end{tabular}
\caption{Paramètres de l'algorithme MODE}
\end{table}

\subsection{Pourquoi MODE pour ce problème ?}

\begin{itemize}
    \item[$\checkmark$] Gère \textbf{plusieurs objectifs} simultanément
    \item[$\checkmark$] Fonctionne avec des \textbf{variables continues} (puissances réelles)
    \item[$\checkmark$] Gère les \textbf{contraintes} (SoC, puissance site)
    \item[$\checkmark$] Pas besoin de dérivées (problème complexe)
    \item[$\checkmark$] Trouve un \textbf{ensemble de solutions} (pas qu'une seule)
\end{itemize}

\section{Résultats Obtenus}

\subsection{Chiffres clés}

\textbf{Exécution :}
\begin{itemize}
    \item Temps : \textbf{30.71 secondes}
    \item Solutions trouvées : \textbf{100 solutions non-dominées}
    \item \textbf{Toutes} respectent les contraintes
\end{itemize}

\textbf{Qualité des solutions :}
\begin{itemize}
    \item \textbf{Hypervolume = 0.7276} $\rightarrow$ Couvre 73\% de l'espace optimal (excellent)
    \item \textbf{Spacing = 0.0312} $\rightarrow$ Distribution très uniforme (excellent)
\end{itemize}

\subsection{Analyse des 3 objectifs}

\subsubsection{Objectif 1 : Coût}

\begin{table}[H]
\centering
\begin{tabular}{lcc}
\toprule
& \textbf{Valeur} & \textbf{Signification} \\
\midrule
\textbf{Meilleur} & \textbf{-4.50 €} & \textbf{Profit !} (V2G marche) \\
Pire & 54.47 € & Coût maximum \\
Moyen & 19.14 € & Coût moyen \\
Variabilité (CV) & 79\% & Très flexible \\
\bottomrule
\end{tabular}
\caption{Statistiques de l'objectif Coût}
\end{table}

\textbf{Ce qu'on comprend :}
\begin{itemize}
    \item On peut \textbf{gagner de l'argent} avec le V2G (-4.50€)
    \item Grande flexibilité : selon la priorité, le coût varie beaucoup
\end{itemize}

\subsubsection{Objectif 2 : Insatisfaction}

\begin{table}[H]
\centering
\begin{tabular}{lcc}
\toprule
& \textbf{Valeur} & \textbf{Signification} \\
\midrule
\textbf{Meilleure} & \textbf{2.76} & $\sim$9\% d'écart par véhicule \\
Pire & 6.93 & $\sim$23\% d'écart par véhicule \\
Moyenne & 4.03 & $\sim$13\% d'écart par véhicule \\
Variabilité (CV) & 26\% & Bien contrôlé \\
\bottomrule
\end{tabular}
\caption{Statistiques de l'objectif Insatisfaction}
\end{table}

\textbf{Ce qu'on comprend :}
\begin{itemize}
    \item Même dans le pire cas, l'insatisfaction reste acceptable
    \item Objectif le plus "stable" (varie peu)
\end{itemize}

\subsubsection{Objectif 3 : Pic de puissance}

\begin{table}[H]
\centering
\begin{tabular}{lccc}
\toprule
& \textbf{Valeur} & \textbf{\% du max site} & \textbf{Interprétation} \\
\midrule
\textbf{Meilleur} & \textbf{14.07 kW} & 23\% & Excellent lissage \\
Pire & 58.37 kW & 97\% & Proche de la limite \\
Moyen & 32.27 kW & 54\% & Utilisation modérée \\
Variabilité (CV) & 34\% & - & Flexible \\
\bottomrule
\end{tabular}
\caption{Statistiques de l'objectif Pic de puissance}
\end{table}

\textbf{Ce qu'on comprend :}
\begin{itemize}
    \item On peut \textbf{réduire le pic de 76\%} (de 58 à 14 kW !)
    \item Aucune solution ne dépasse la limite de 60 kW
\end{itemize}

\section{Analyse des Métriques de Performance}

\subsection{Hypervolume (HV)}

\subsubsection{Définition}

L'hypervolume mesure le \textbf{volume de l'espace des objectifs dominé} par le front de Pareto, par rapport à un point de référence.

\begin{equation}
HV = \text{Volume}\left(\{y \in \mathbb{R}^3 \mid \exists x \in PF : y \preceq f(x) \preceq r\}\right)
\end{equation}

où :
\begin{itemize}
    \item $PF$ : Front de Pareto (100 solutions)
    \item $f(x)$ : Valeurs des objectifs pour la solution $x$
    \item $r$ : Point de référence = (1, 1, 1) après normalisation
    \item $\preceq$ : Relation de dominance
\end{itemize}

\subsubsection{Calcul dans notre cas}

\textbf{Étape 1 : Normalisation}

Chaque objectif est normalisé entre 0 (meilleur) et 1 (pire) :

\begin{align}
\text{Coût}_{\text{norm}} &= \frac{\text{Coût} - (-4.50)}{54.47 - (-4.50)} = \frac{\text{Coût} + 4.50}{58.97} \\
\text{Insatis}_{\text{norm}} &= \frac{\text{Insatis} - 2.76}{6.93 - 2.76} = \frac{\text{Insatis} - 2.76}{4.17} \\
\text{Pic}_{\text{norm}} &= \frac{\text{Pic} - 14.07}{58.37 - 14.07} = \frac{\text{Pic} - 14.07}{44.30}
\end{align}

\textbf{Étape 2 : Calcul de l'hypervolume}

Algorithme WFG (While et al., 2006) :
\begin{itemize}
    \item Décompose l'espace en hypercubes
    \item Somme les volumes non-dominés
\end{itemize}

\textbf{Résultat : HV = 0.7276}

\subsubsection{Interprétation}

\textbf{Notre HV = 0.7276 signifie que :}
\begin{itemize}
    \item \textbf{72.76\% de l'espace optimal} est couvert par nos 100 solutions
    \item L'algorithme a bien convergé vers le front de Pareto optimal
    \item Grande diversité de compromis disponibles pour le décideur
\end{itemize}

\textbf{Pourquoi pas 100\% ?}
\begin{itemize}
    \item Les trois objectifs sont contradictoires : impossible d'optimiser tous simultanément
    \item Les contraintes physiques limitent les solutions possibles (SoC entre 0-100\%, puissance max 60kW)
    \item Le point de référence (1,1,1) représente le "pire cas" sur tous les objectifs combinés
\end{itemize}

\textbf{Échelle de qualité :}
\begin{itemize}
    \item HV $<$ 0.4 : Faible
    \item HV 0.4-0.5 : Moyenne
    \item HV 0.5-0.6 : Bonne
    \item HV 0.6-0.7 : Très bonne
    \item \textbf{HV 0.7-0.8 : Excellente} $\leftarrow$ \textbf{Notre résultat}
    \item HV $>$ 0.8 : Exceptionnelle
\end{itemize}

\subsection{Spacing (SP)}

\subsubsection{Définition}

Le spacing mesure l'\textbf{uniformité de la distribution} des solutions sur le front de Pareto.

\textbf{Formule (Schott, 1995) :}
\begin{equation}
SP = \sqrt{\frac{1}{N-1} \sum_{i=1}^{N} (d_i - \bar{d})^2}
\end{equation}

où :
\begin{itemize}
    \item $N = 100$ (nombre de solutions)
    \item $d_i$ = distance minimale entre la solution $i$ et ses voisines
    \item $\bar{d}$ = moyenne des distances
\end{itemize}

\textbf{Calcul de $d_i$ :}
\begin{equation}
d_i = \min_{j \neq i} \|f(x_i) - f(x_j)\|_2
\end{equation}

\subsubsection{Interprétation}

\textbf{Notre SP = 0.0312 signifie que :}
\begin{itemize}
    \item Les 100 solutions sont \textbf{très uniformément réparties} sur le front de Pareto
    \item Pas de "trous" ou zones sous-représentées
    \item Écart-type des distances entre solutions voisines est très faible
    \item Le décideur dispose de solutions bien distribuées pour explorer tous les compromis
\end{itemize}

\textbf{Valeur de référence :}
\begin{itemize}
    \item \textbf{SP $<$ 0.05 : Excellent} $\leftarrow$ \textbf{Notre résultat}
    \item SP 0.05-0.10 : Très bon
    \item SP 0.10-0.20 : Bon
    \item SP 0.20-0.30 : Moyen
    \item SP $>$ 0.30 : Faible
\end{itemize}

\textbf{Visualisation conceptuelle :}

\begin{verbatim}
Mauvais spacing (SP > 0.2)     Notre spacing (SP = 0.031)
    •                               •  •  •  •  •
    • •                             •  •  •  •  •
         • •                        •  •  •  •  •
                  •                 •  •  •  •  •
\end{verbatim}

\subsection{Synthèse des métriques}

\begin{table}[H]
\centering
\begin{tabular}{lccc}
\toprule
\textbf{Critère} & \textbf{Valeur} & \textbf{Interprétation} & \textbf{Qualité} \\
\midrule
Hypervolume & 0.7276 & Couvre 73\% de l'espace & \textcolor{green}{Excellente} \\
Spacing & 0.0312 & Distribution uniforme & \textcolor{green}{Excellente} \\
Nb solutions & 100 & Large choix & \textcolor{green}{Très bon} \\
Temps exec. & 34 s & Rapide & \textcolor{green}{Très bon} \\
Contraintes & 100\% & Toutes respectées & \textcolor{green}{Parfait} \\
\bottomrule
\end{tabular}
\caption{Synthèse des performances de MODE}
\end{table}

\textbf{Bilan :} L'algorithme MODE a produit des résultats de très bonne qualité pour ce problème d'optimisation de charge de véhicules électriques.

\section{Ce qu'on a compris}

\subsection{Sur le problème multi-objectifs}

\textbf{Constat principal :} Les 3 objectifs sont \textbf{vraiment contradictoires}

Exemples concrets :
\begin{itemize}
    \item Pour gagner de l'argent (coût -4.50€) $\rightarrow$ Il faut utiliser le V2G $\rightarrow$ Ça crée des pics
    \item Pour avoir un petit pic (14 kW) $\rightarrow$ Il faut charger lentement $\rightarrow$ Ça insatisfait les utilisateurs
    \item Pour satisfaire tout le monde $\rightarrow$ Il faut charger vite $\rightarrow$ Ça coûte cher et crée des pics
\end{itemize}

\textbf{Conclusion :} C'est normal qu'il n'y ait pas UNE solution optimale. On doit \textbf{choisir selon la priorité}.

\subsection{Sur le tarif TOU}

Le prix qui varie selon l'heure change beaucoup les choses :

\begin{enumerate}
    \item \textbf{Incitation naturelle :} MODE privilégie les heures creuses pour charger
    \item \textbf{V2G rentable :} Décharger aux heures de pointe (0.30 \$/kWh) génère des revenus
    \item \textbf{Compromis nécessaire :} Charger uniquement la nuit insatisfait ceux qui partent tôt le matin
\end{enumerate}

\subsection{Sur MODE}

\textbf{Ce qu'on observe :}
\begin{itemize}
    \item MODE \textbf{converge bien} : HV = 0.73 (excellent)
    \item Les solutions sont \textbf{bien réparties} : SP = 0.031 (très uniforme)
    \item C'est \textbf{rapide} : 34 secondes pour 720 variables
    \item \textbf{1500 générations} suffisent pour explorer correctement l'espace
\end{itemize}

\subsection{Validation}

\textbf{Pourquoi on peut faire confiance aux résultats :}

\begin{itemize}
    \item[$\checkmark$] \textbf{Contraintes respectées :} 100\% des solutions sont valides
    \begin{itemize}
        \item SoC toujours entre 0\% et 100\%
        \item Puissance site jamais $>$ 60 kW
        \item Respect de la disponibilité des véhicules
    \end{itemize}

    \item[$\checkmark$] \textbf{Cohérence physique :}
    \begin{itemize}
        \item Quand coût $<$ 0 $\rightarrow$ On voit bien des flux négatifs (V2G)
        \item Charge rapide $\rightarrow$ Pics élevés
        \item Charge lente $\rightarrow$ Pics faibles
    \end{itemize}

    \item[$\checkmark$] \textbf{Reproductible :}
    \begin{itemize}
        \item Avec le même seed (seed=1), on obtient toujours les mêmes résultats
    \end{itemize}
\end{itemize}

\section{Limites et Améliorations Possibles}

\subsection{Ce qu'on n'a pas fait (limites)}

\textbf{Sur le modèle :}
\begin{itemize}
    \item Tarifs TOU fixes (pas de prévision de prix futurs)
    \item Batterie simplifiée (pas de dégradation, rendement 100\%)
    \item On suppose qu'on connaît les heures d'arrivée/départ à l'avance
    \item Optimisation sur 24h seulement (pas multi-jours)
\end{itemize}

\textbf{Sur l'algorithme :}
\begin{itemize}
    \item Paramètres F et CR fixés manuellement
    \item Une seule exécution (pas de statistiques sur plusieurs runs)
\end{itemize}

\subsection{Ce qu'on pourrait améliorer}

\textbf{Court terme :}
\begin{itemize}
    \item Tester différentes valeurs de F et CR
    \item Faire plusieurs exécutions et calculer les moyennes
    \item Ajouter un modèle de dégradation de batterie
\end{itemize}

\textbf{Long terme :}
\begin{itemize}
    \item Prévoir les prix futurs de l'électricité
    \item Optimiser sur plusieurs jours
    \item Prendre en compte l'incertitude sur les arrivées/départs
\end{itemize}

\section{Conclusion}

\subsection{Ce qu'on a réalisé}

\begin{itemize}
    \item[$\checkmark$] Implémenté MODE pour optimiser 720 variables
    \item[$\checkmark$] Utilisé des \textbf{données réelles} (Caltech, 30 véhicules)
    \item[$\checkmark$] Obtenu \textbf{100 solutions de qualité} (HV=0.73, SP=0.031)
    \item[$\checkmark$] Prouvé la \textbf{viabilité du V2G} (jusqu'à -4.50€ de profit)
    \item[$\checkmark$] Montré qu'on peut \textbf{réduire le pic de 76\%} (de 58 à 14 kW)
\end{itemize}

\subsection{Réponse à la question initiale}

\textbf{Question :} Comment optimiser le coût, la satisfaction et le pic simultanément ?

\textbf{Réponse :}
On \textbf{ne peut pas} avoir UNE solution qui optimise tout. Mais on peut trouver \textbf{100 solutions non-dominées} qui offrent différents compromis :

\begin{table}[H]
\centering
\begin{tabular}{lcc}
\toprule
\textbf{Priorité} & \textbf{Choix de solution} & \textbf{Résultat} \\
\midrule
Gagner de l'argent & Solution à coût minimal & -4.50€ (profit V2G) \\
Satisfaire les clients & Solution à insatisfaction minimale & Écart de 9\% seulement \\
Protéger le réseau & Solution à pic minimal & 14 kW (au lieu de 58) \\
\bottomrule
\end{tabular}
\caption{Exemples de choix selon les priorités}
\end{table}

\textbf{Le gestionnaire choisit selon sa priorité du moment.}

\subsection{Ce qu'on a appris}

\textbf{Sur le multi-objectifs :}
\begin{itemize}
    \item Les conflits entre objectifs sont réels et inévitables
    \item C'est normal de ne pas pouvoir tout optimiser
    \item Le front de Pareto donne le \textbf{choix} au décideur
\end{itemize}

\textbf{Sur MODE :}
\begin{itemize}
    \item Efficace pour ce type de problème (73\% de l'espace couvert)
    \item Converge rapidement (34 secondes)
    \item Produit des solutions bien distribuées
\end{itemize}

\textbf{Sur le V2G :}
\begin{itemize}
    \item C'est rentable avec un tarif TOU
    \item Maximiser le profit crée des pics
    \item Il faut trouver le bon équilibre
\end{itemize}

\section*{Références}

\begin{itemize}
    \item \textbf{Qian, J., Jiang, Y., Liu, X., Wang, Q., Wang, T., Shi, Y., \& Chen, W. (2023).} Federated Reinforcement Learning for Electric Vehicles Charging Control on Distribution Networks. \textit{arXiv preprint arXiv:2308.08792}. \label{qian2023federated}

    \item \textbf{Storn, R., \& Price, K. (1997).} Differential evolution--a simple and efficient heuristic for global optimization over continuous spaces. \textit{Journal of global optimization}, 11(4), 341-359.

    \item \textbf{Deb, K., Pratap, A., Agarwal, S., \& Meyarivan, T. A. M. T. (2002).} A fast and elitist multiobjective genetic algorithm: NSGA-II. \textit{IEEE transactions on evolutionary computation}, 6(2), 182-197.

    \item \textbf{While, L., Hingston, P., Barone, L., \& Huband, S. (2006).} A faster algorithm for calculating hypervolume. \textit{IEEE transactions on evolutionary computation}, 10(1), 29-38.

    \item \textbf{Schott, J. R. (1995).} Fault tolerant design using single and multicriteria genetic algorithm optimization. PhD thesis, Massachusetts Institute of Technology.

    \item \textbf{Caltech ACN-Data :} \url{https://ev.caltech.edu/}
\end{itemize}

\end{document}
